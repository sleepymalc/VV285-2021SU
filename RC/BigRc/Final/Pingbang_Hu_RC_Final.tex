\documentclass[10pt, t, allowdisplaybreaks]{beamer}
\usepackage{amsmath}
\usepackage{setspace}
\usepackage{float} 
\usepackage{multido}
\usepackage{multirow}
\usepackage{array}
\usepackage{enumerate}
\usepackage{booktabs}
\usepackage{indentfirst} 
\usepackage[style=mla]{biblatex}
\usepackage{setspace}
\usepackage{calligra}
\usepackage{subcaption}
\usepackage{hyperref}
\usepackage{textpos}

\makeatletter
\let\@@magyar@captionfix\relax
\makeatother

\definecolor{Turquoise3}{RGB}{0, 134, 139}
\renewcommand{\emph}[1]{{\color{Turquoise3}\textsl{#1}}}
\newcommand{\C}{\mathbb{C}} \newcommand{\F}{\mathbb{F}} \newcommand{\R}{\mathbb{R}} \newcommand{\Q}{\mathbb{Q}}
\newcommand{\N}{\mathbb{N}}
\newcommand{\myseries}[2]{$#1_1,#1_2,\dots,#1_#2$}
\newcommand{\nullspace}{~\\[15pt]}
\newcommand{\rot}{\text{\Large{\calligra{rot}}\,}}
\newcommand{\Remark}{\textbf{Remark: }}
\newcommand{\Question}{\textbf{Question: }}
\newcommand{\Extension}{\textbf{Extension: }}
\newcommand{\scp}[2]{\langle\,#1\,,\,#2\,\rangle} \newcommand{\scpp}{\langle\,\cdot\,,\,\cdot\,\rangle}
\newcommand{\myqed}{\hfill$\blacksquare$}
\newcommand{\nullspacesmall}{~\vspace{1em}}
\newcommand{\nullspacemid}{~\vspace{4em}}
\newcommand{\nullspacebig}{~\vspace{6em}}
\newcommand{\at}[3]{\left.#1\right\vert_{#2}^{#3}}

\usetheme{Madrid}
\setbeamertemplate{navigation symbols}{}

\addtobeamertemplate{frametitle}{}{
\begin{textblock*}{100mm}(0.85\textwidth,-1cm)
\includegraphics[height=1cm]{Figures/logo/logo.png}
\end{textblock*}}

\definecolor{themecolor}{RGB}{25,25,112} 

\usecolortheme[named=themecolor]{structure}

\setbeamertemplate{items}[default]

\hypersetup{
    colorlinks=true,
    linkcolor=themecolor,
    filecolor=themecolor,      
    urlcolor=themecolor,
    citecolor=themecolor,
}

\title{VV285 RC Final}
\subtitle{\textbf{Final Exercise}\\\large Last but not least\dots}
\institute[UM-SJTU JI]{University of Michigan-Shanghai Jiao Tong University Joint Institute}
\author{Pingbang Hu}

\begin{document}

\begin{frame}
    \titlepage
    \begin{center}
        \includegraphics[height=2cm]{Figures/logo/logo2.png}
    \end{center}
\end{frame}

\begin{frame}
    \frametitle{Exercise 1}
    \par Calculate the directional derivative of the continuous function
    \begin{equation*}
        f:\R^2\rightarrow R, \qquad f(x,y) = \sqrt[3]{x^2y}
    \end{equation*}
    at $(x,y) = (0,0)$ in the direction of the vector $h = (\frac{1}{\sqrt{2}}, \frac{1}{\sqrt{2}})$.
\end{frame}


\begin{frame}
    \frametitle{Solution 1}
    \par Calculate the directional derivative of the continuous function
    \begin{equation*}
        f:\R^2\rightarrow R, \qquad f(x,y) = \sqrt[3]{x^2y}
    \end{equation*}
    at $(x,y) = (0,0)$ in the direction of the vector $h = (\frac{1}{\sqrt{2}}, \frac{1}{\sqrt{2}})$.

    \nullspacesmall

    \par From the definition, we need to calculate \dots
\end{frame}

\begin{frame}
    \frametitle{Solution 1}
    \par Calculate the directional derivative of the continuous function
    \begin{equation*}
        f:\R^2\rightarrow R, \qquad f(x,y) = \sqrt[3]{x^2y}
    \end{equation*}
    at $(x,y) = (0,0)$ in the direction of the vector $h = (\frac{1}{\sqrt{2}}, \frac{1}{\sqrt{2}})$.

    \nullspacesmall

    \par From the definition, we need to calculate
    \begin{equation*}
        \frac{d}{dt} \at{f(x_0 + th)}{t = 0}{}
    \end{equation*}
    where $x_0 = 0$ and $h = (\frac{1}{\sqrt{2}}, \frac{1}{\sqrt{2}})$.
\end{frame}

\begin{frame}[allowframebreaks]
    \frametitle{Solution 1}
    \par Calculate the directional derivative of the continuous function
    \begin{equation*}
        f:\R^2\rightarrow R, \qquad f(x,y) = \sqrt[3]{x^2y}
    \end{equation*}
    at $(x,y) = (0,0)$ in the direction of the vector $h = (\frac{1}{\sqrt{2}}, \frac{1}{\sqrt{2}})$.

    \nullspacesmall

    \par From the definition, we need to calculate
    \begin{equation*}
        \frac{d}{dt} \at{f(x_0 + th)}{t = 0}{}
    \end{equation*}
    where $x_0 = 0$ and $h = (\frac{1}{\sqrt{2}}, \frac{1}{\sqrt{2}})$. Hence, we have 
    \begin{equation*}
        f(x_0 + th) = f(t\cdot \begin{pmatrix}
            1/\sqrt{2}\\
            1\sqrt{2}
        \end{pmatrix}) = \sqrt[3]{\frac{t^2}{2}\cdot \frac{t}{\sqrt{2}}} = \frac{t}{\sqrt{2}}
    \end{equation*}
    so that 

\end{frame}

\begin{frame}
    \frametitle{Solution 1}
    \par Calculate the directional derivative of the continuous function
    \begin{equation*}
        f:\R^2\rightarrow R, \qquad f(x,y) = \sqrt[3]{x^2y}
    \end{equation*}
    at $(x,y) = (0,0)$ in the direction of the vector $h = (\frac{1}{\sqrt{2}}, \frac{1}{\sqrt{2}})$.

    \nullspacesmall

    \par From the definition, we need to calculate
    \begin{equation*}
        \frac{d}{dt} \at{f(x_0 + th)}{t = 0}{}
    \end{equation*}
    where $x_0 = 0$ and $h = (\frac{1}{\sqrt{2}}, \frac{1}{\sqrt{2}})$. Hence, we have 
    \begin{equation*}
        f(x_0 + th) = f(t\cdot \begin{pmatrix}
            1/\sqrt{2}\\
            1\sqrt{2}
        \end{pmatrix}) = \sqrt[3]{\frac{t^2}{2}\cdot \frac{t}{\sqrt{2}}} = \frac{t}{\sqrt{2}}
    \end{equation*}
    so that 
    \begin{equation*}
        \frac{d}{dt}f(x_0 + th) = \frac{d}{dt}\frac{t}{\sqrt{2}} = \frac{1}{\sqrt{2}}.
    \end{equation*}
    
    \myqed

\end{frame}


\begin{frame}
    \frametitle{Exercise 2}
    \par Let $g:(0, \infty)\rightarrow \R$ be a differentiable function and let 
    $\left\lVert x\right\rVert = \sqrt{x^2_1+x^2_2+x^2_3}$ for $x = (x_1, x_2, x_3)\in \R^3$.
    \par Prove that the vector field 
    \begin{equation*}
        F:\R^3\backslash\{0\}\rightarrow \R^3, \qquad F(x) = g(\left\lVert x\right\rVert )x
    \end{equation*}
    is conservative.
\end{frame}

\begin{frame}
    \frametitle{Solution 2}
    \par Let $g:(0, \infty)\rightarrow \R$ be a differentiable function and let 
    $\left\lVert x\right\rVert = \sqrt{x^2_1+x^2_2+x^2_3}$ for $x = (x_1, x_2, x_3)\in \R^3$.
    \par Prove that the vector field 
    \begin{equation*}
        F:\R^3\backslash\{0\}\rightarrow \R^3, \qquad F(x) = g(\left\lVert x\right\rVert )x
    \end{equation*}
    is conservative.
    
    \nullspacesmall

    \par The set $\R^3\backslash\{0\}$ is simply connected
\end{frame}

\begin{frame}
    \frametitle{Solution 2}
    \par Let $g:(0, \infty)\rightarrow \R$ be a differentiable function and let 
    $\left\lVert x\right\rVert = \sqrt{x^2_1+x^2_2+x^2_3}$ for $x = (x_1, x_2, x_3)\in \R^3$.
    \par Prove that the vector field 
    \begin{equation*}
        F:\R^3\backslash\{0\}\rightarrow \R^3, \qquad F(x) = g(\left\lVert x\right\rVert )x
    \end{equation*}
    is conservative.
    
    \nullspacesmall

    \par The set $\R^3\backslash\{0\}$ is simply connected, so it suffices to show that $\text{rot} F = 0$. 
\end{frame}

\begin{frame}
    \frametitle{Solution 2}
    \par Let $g:(0, \infty)\rightarrow \R$ be a differentiable function and let 
    $\left\lVert x\right\rVert = \sqrt{x^2_1+x^2_2+x^2_3}$ for $x = (x_1, x_2, x_3)\in \R^3$.
    \par Prove that the vector field 
    \begin{equation*}
        F:\R^3\backslash\{0\}\rightarrow \R^3, \qquad F(x) = g(\left\lVert x\right\rVert )x
    \end{equation*}
    is conservative.
    
    \nullspacesmall

    \par The set $\R^3\backslash\{0\}$ is simply connected, so it suffices to show that $\text{rot} F = 0$. Now 
    \begin{equation*}
        \left\lvert (\text{rot}F)_i\right\rvert = \left\lvert \frac{\partial F_j}{\partial x_k} - \frac{\partial F_k}{\partial x_j}\right\rvert 
    \end{equation*}
    where $(i,j,k)$ is any one of the permutation of $\{1,2,3\}$. 
\end{frame}


\begin{frame}
    \frametitle{Solution 2}
    \par Let $g:(0, \infty)\rightarrow \R$ be a differentiable function and let 
    $\left\lVert x\right\rVert = \sqrt{x^2_1+x^2_2+x^2_3}$ for $x = (x_1, x_2, x_3)\in \R^3$.
    \par Prove that the vector field 
    \begin{equation*}
        F:\R^3\backslash\{0\}\rightarrow \R^3, \qquad F(x) = g(\left\lVert x\right\rVert )x
    \end{equation*}
    is conservative.
    
    \nullspacesmall

    \par The set $\R^3\backslash\{0\}$ is simply connected, so it suffices to show that $\text{rot} F = 0$. Now 
    \begin{equation*}
        \left\lvert (\text{rot}F)_i\right\rvert = \left\lvert \frac{\partial F_j}{\partial x_k} - \frac{\partial F_k}{\partial x_j}\right\rvert 
    \end{equation*}
    where $(i,j,k)$ is any one of the permutation of $\{1,2,3\}$. 
\end{frame}

\begin{frame}
    \frametitle{Solution 2}
    We then have 
    \begin{equation*}
        \begin{split}
            \frac{\partial F_j}{\partial x_k} &= \frac{\partial}{\partial x_k}\left(g(\left\lVert x\right\rVert )x\right)_j
            = \frac{\partial}{\partial x_k}g(\left\lVert x\right\rVert )x_j\\\\
        \end{split}
    \end{equation*}
\end{frame}

\begin{frame}
    \frametitle{Solution 2}
    We then have 
    \begin{equation*}
        \begin{split}
            \frac{\partial F_j}{\partial x_k} &= \frac{\partial}{\partial x_k}\left(g(\left\lVert x\right\rVert )x\right)_j
            = \frac{\partial}{\partial x_k}g(\left\lVert x\right\rVert )x_j\\\\
            &= x_j g'(\left\lVert x\right\rVert )\frac{\partial }{\partial x_k}\sqrt{x_1^2+x_2^2+x_3^2} \\\\
        \end{split}
    \end{equation*}
\end{frame}
\begin{frame}
    \frametitle{Solution 2}
    We then have 
    \begin{equation*}
        \begin{split}
            \frac{\partial F_j}{\partial x_k} &= \frac{\partial}{\partial x_k}\left(g(\left\lVert x\right\rVert )x\right)_j
            = \frac{\partial}{\partial x_k}g(\left\lVert x\right\rVert )x_j\\\\
            &= x_j g'(\left\lVert x\right\rVert )\frac{\partial }{\partial x_k}\sqrt{x_1^2+x_2^2+x_3^2} \\\\
            &= x_j g'(\left\lVert x\right\rVert )\frac{x_k}{\sqrt{x_1^2+x_2^2+x_3^2}} \\\\
        \end{split}
    \end{equation*}
\end{frame}
\begin{frame}
    \frametitle{Solution 2}
    We then have 
    \begin{equation*}
        \begin{split}
            \frac{\partial F_j}{\partial x_k} &= \frac{\partial}{\partial x_k}\left(g(\left\lVert x\right\rVert )x\right)_j
            = \frac{\partial}{\partial x_k}g(\left\lVert x\right\rVert )x_j\\\\
            &= x_j g'(\left\lVert x\right\rVert )\frac{\partial }{\partial x_k}\sqrt{x_1^2+x_2^2+x_3^2} \\\\
            &= x_j g'(\left\lVert x\right\rVert )\frac{x_k}{\sqrt{x_1^2+x_2^2+x_3^2}} \\\\
            &= x_j x_k \frac{g'(\left\lVert x\right\rVert )}{\left\lVert x\right\rVert }\\\\
        \end{split}
    \end{equation*}
\end{frame}
\begin{frame}
    \frametitle{Solution 2}
    We then have 
    \begin{equation*}
        \begin{split}
            \frac{\partial F_j}{\partial x_k} &= \frac{\partial}{\partial x_k}\left(g(\left\lVert x\right\rVert )x\right)_j
            = \frac{\partial}{\partial x_k}g(\left\lVert x\right\rVert )x_j\\\\
            &= x_j g'(\left\lVert x\right\rVert )\frac{\partial }{\partial x_k}\sqrt{x_1^2+x_2^2+x_3^2} \\\\
            &= x_j g'(\left\lVert x\right\rVert )\frac{x_k}{\sqrt{x_1^2+x_2^2+x_3^2}} \\\\
            &= x_j x_k \frac{g'(\left\lVert x\right\rVert )}{\left\lVert x\right\rVert } = \frac{\partial F_k}{\partial x_j}
        \end{split}
    \end{equation*}
\end{frame}
\begin{frame}
    \frametitle{Solution 2}
    We then have 
    \begin{equation*}
        \begin{split}
            \frac{\partial F_j}{\partial x_k} &= \frac{\partial}{\partial x_k}\left(g(\left\lVert x\right\rVert )x\right)_j
            = \frac{\partial}{\partial x_k}g(\left\lVert x\right\rVert )x_j\\\\
            &= x_j g'(\left\lVert x\right\rVert )\frac{\partial }{\partial x_k}\sqrt{x_1^2+x_2^2+x_3^2} \\\\
            &= x_j g'(\left\lVert x\right\rVert )\frac{x_k}{\sqrt{x_1^2+x_2^2+x_3^2}} \\\\
            &= x_j x_k \frac{g'(\left\lVert x\right\rVert )}{\left\lVert x\right\rVert } = \frac{\partial F_k}{\partial x_j}
        \end{split}
    \end{equation*}

    \nullspacesmall

    \par Hence, the rotation of $F$ vanishes everywhere on $\R^3\backslash\{0\}$.

    \myqed

\end{frame}

\begin{frame}
    \frametitle{Exercise 3}

\end{frame}

\begin{frame}
    \frametitle{Exercise 4}

\end{frame}

\begin{frame}
    \frametitle{Exercise 5}

\end{frame}
\end{document}